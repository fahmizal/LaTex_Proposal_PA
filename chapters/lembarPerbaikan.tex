% Bagian Header
\noindent



\begin{table}[H]
    \begin{tabular}{lll}
       Nama  & : \penulis & \\
       NIM   & : \nim & \\
       Prodi   & : \prodi & \\
       Judul PA   & : \judulid & \\
       Waktu Pendadaran   & : Tanggal/bulan/tahun & \\
    \end{tabular}
\end{table}

\vspace{0.5cm}

% Tabel
\setlength{\arrayrulewidth}{0.5mm}
\setlength{\tabcolsep}{10pt}
\renewcommand{\arraystretch}{1.5}

\begin{longtable}{|m{3cm}|m{1cm}|m{3.4cm}|m{2.4cm}|m{2cm}|}
    \hline
    \textbf{Dosen Penguji} & \textbf{BAB} & \textbf{Saran Perbaikan} & \textbf{Keterangan} & \textbf{Halaman Perbaikan} \\
    \hline
    & & & & \\ [1cm] % Baris pertama
    \hline
    & & & & \\ [1cm] % Baris kedua
    \hline
    & & & & \\ [1cm] % Baris ketiga
    \hline
    & & & & \\ [1cm] % Baris keempat
    \hline
    & & & & \\ [1cm] % Baris kelima
    \hline
    & & & & \\ [1cm] % Baris kelima
    \hline
    & & & & \\ [1cm] % Baris kelima
    \hline
    & & & & \\ [1cm] % Baris kelima
    \hline
\end{longtable}
