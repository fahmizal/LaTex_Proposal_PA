\chapter[LANDASAN TEORI]{\\ LANDASAN TEORI}

\section{Tinjauan Pustaka}
\lipsum[1]

\begin{table}[H]
    \centering
    \caption{Contoh Tabel 1}
    \label{t risetPemodelan}
    \begin{tabularx}{\linewidth}{
        |p{\dimexpr.27\linewidth-2\tabcolsep-1.3333\arrayrulewidth}% column 1
        |p{\dimexpr.33\linewidth-2\tabcolsep-1.3333\arrayrulewidth}% column 2
        |p{\dimexpr.40\linewidth-2\tabcolsep-1.3333\arrayrulewidth}|% column 3
    }
        \hline
        Penulis & Judul & Metode Pemodelan Sistem\\ \hline
        Penulis 1 & Judul 1 & Pemodelan Sistem 1 \\ \hline
        Penulis 2 & Judul 2 & Pemodelan Sistem 2 \\ \hline
        Penulis 3 & Judul 3 & Pemodelan Sistem 3 \\ \hline
        Penulis 4 & Judul 4 & Pemodelan Sistem 4 \\ \hline
        Penulis 5 & Judul 5 & Pemodelan Sistem 5 \\ \hline
        Penulis 6 & Judul 6 & Pemodelan Sistem 6 \\ \hline
    \end{tabularx}
\end{table}

\begin{table}[H]
    \centering
    \caption{Contoh Tabel 2}
    \label{t blok}
    \begin{tabular}{|c|l|l|}
        \hline
        \rowcolor[HTML]{C0C0C0} 
        {\color[HTML]{000000} No.} & \multicolumn{1}{c|}{\cellcolor[HTML]{C0C0C0}{\color[HTML]{000000} Nama}} & \multicolumn{1}{c|}{\cellcolor[HTML]{C0C0C0}Fungsi}  \\ \hline
        \rowcolor[HTML]{FFFFFF} 
        1 & \textit{Nama rincian objek yang akan dibahas 1} & Fungsi rincian objek penjelasan 1 \\ \hline
        2 & \textit{Nama rincian objek yang akan dibahas 2} & Fungsi rincian objek penjelasan 2 \\ \hline
        3 & \textit{Nama rincian objek yang akan dibahas 3} & Fungsi rincian objek penjelasan 3 \\ \hline
        4 & \textit{Nama rincian objek yang akan dibahas 4} & Fungsi rincian objek penjelasan 4 \\ \hline
    \end{tabular}
\end{table}

\section{Dasar Teori}

\subsection{Materi Penjelasan Dasar Teori 1}
\lipsum[1]

\subsection{Materi Penjelasan Dasar Teori 2}
\subsubsection{Sub Materi Penjelasan 2}
\lipsum[1]

\begin{equation}
	\label{eq tforigin}
	\begin{split}
		Tf = G(s) = & \frac{output}{input} \\
		Tf= & \frac{b_{0}s^{m} + b_{1}s^{m-1} + ... + b_{m-1}s + b_{m}}{a_{0}s^{n} + a_{1}s^{n-1} + ... + a_{n-1}s + a_{n}}
	\end{split}
\end{equation}

\lipsum[1]

\subsubsection{Sub Materi Penjelasan 2}
\lipsum[1]
\begin{enumerate}
    \item Penjelasan poin 1
    \item Penjelasan poin 2
    \item Penjelasan poin 3
    \item Penjelasan poin 4
\end{enumerate}

\subsubsection{Sub Materi Penjelasan 3}
\lipsum[1]

\begin{equation}
	\label{eq liniernaik}
	linear(x; a,b) =
	\left\{\begin{matrix}
		0; \; x \leq a\\ 
		(x-a)/(b-a); \; a \leq x \leq b\\ 
		1; \; x \geq b
	\end{matrix}\right.
\end{equation}

\lipsum[1]