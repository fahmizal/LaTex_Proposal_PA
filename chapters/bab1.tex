\chapter[PENDAHULUAN]{\\ PENDAHULUAN}

\section{Latar Belakang}

\lipsum[1]

\section{Rumusan Masalah}
Berikut adalah beberapa poin yang telah ditemukan dan dirumuskan sebagai masalah utama yang akan menjadi fokus penelitian ini \cite{chan2013review}:
\begin{enumerate}
    \item Rumusan masalah 1
    \item Rumusan masalah 2
    \item Rumusan masalah 3	
\end{enumerate}

\section{Batasan Masalah}
Agar penelitian tetap terfokus pada ruang lingkup yang telah ditetapkan \cite{muhammadred}, akan dijelaskan batasan-batasan yang perlu dipertimbangkan dalam pengembangan penelitian ini sebagai berikut:
\begin{enumerate}
    \item Batasan masalah 1
    \item Batasan masalah 2
    \item Batasan masalah 3
\end{enumerate}

\section{Tujuan Penelitian}
Berikut adalah beberapa tujuan penelitian yang telah ditetapkan untuk memandu jalannya penelitian ini: 
\begin{enumerate}
    \item Tujuan Penelitian 1
    \item Tujuan Penelitian 2
    \item Tujuan Penelitian 3
\end{enumerate}

\section{Manfaat Penelitian}
Adapun manfaat-manfaat yang diperoleh dari penelitian ini, yaitu:
\begin{enumerate}
    \item Manfaat Penelitian 1
    \item Manfaat Penelitian 2
    \item Manfaat Penelitian 3
\end{enumerate}

\section{Sistematika Penulisan}
Laporan proyek akhir ditulis dengan sistematika penulisan sebagai berikut.
\begin{enumerate}
    \item Bab I Pendahuluan: menjelaskan latar belakang, rumusan masalah, tujuan, batasan masalah, serta sistematika penulisan dari laporan proyek akhir sarjana terapan.
    \item Bab II Tinjauan Pustaka: menjabarkan tentang studi yang telah dilakukan oleh peneliti sebelumnya yang berhubungan dengan topik yang diteliti dalam proyek akhir sarjana terapan, serta membahas teori yang relevan dengan masalah yang akan diteliti. Bab ini berisi tentang kajian pustaka yang diperoleh dari berbagai sumber yang terkait dengan masalah yang akan diteliti.
    \item Bab III Metodologi Penelitian: menjabarkan tentang langkah-langkah dan desain yang dilakukan dalam penelitian, serta metode yang digunakan untuk pendekatan dalam penelitian ini.
    \item Bab IV Hasil dan Pembahasan: menjabarkan hasil yang diperoleh dari proyek akhir sarjana terapan dan memberikan pembahasan yang mendalam terkait dengan hasil tersebut. Bab ini juga berisi tentang interpretasi data yang diperoleh dari penelitian.
    \item Bab V Kesimpulan dan Saran: menjabarkan kesimpulan yang diperoleh dari proyek akhir sarjana terapan serta saran yang diberikan untuk penelitian selanjutnya.
    \item Daftar Pustaka: menjabarkan sumber-sumber yang digunakan dalam laporan proyek akhir sarjana terapan.
\end{enumerate}

Secara keseluruhan, sistematika penulisan dalam laporan proyek akhir sarjana terapan adalah susunan atau struktur dari laporan proyek akhir sarjana terapan yang menjabarkan bagian-bagian yang harus ada dalam laporan proyek akhir sarjana terapan, yang meliputi Pendahuluan, Tinjauan Pustaka, Metode Penelitian, Hasil dan Pembahasan, Kesimpulan dan Saran, serta Daftar Pustaka. Sistematika penulisan yang baik akan membuat laporan proyek akhir sarjana terapan lebih mudah untuk dibaca dan dipahami.